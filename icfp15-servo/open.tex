%!TEX root = paper.tex

\section{Open problems}
\label{sec:open}
While this work has discussed many challenges in browser design and our current progress,
there are many other interesting open problems.

\paragraph{Just-in-time code.} JavaScript engines dynamically produce native code that is
intended to execute more efficiently than an interpreted strategy.
Unfortunately, this area is a large source of security bugs.
These bugs come from two sources.
First, there are potential correctness issues.
Many of these optimizations are only valid when certain conditions of the calling
code and environment hold, and ensuring the specialized code is called only when those
conditions hold is non-trivial.
Second, dynamically producing and compiling native code and patching it into memory
while respecting all of the invariants required by the JavaScript runtime (e.g., the
garbage collector's read/write barriers or free vs. in-use registers) is also a challenge.

\paragraph{Integer overflow/underflow.} It is still an open problem
to provide optimized code that checks for overflow or underflow without
incurring significant performance penalties.
The current plan for Rust is to have debug-only checking of integer ranges
and for Servo to run debug builds against a test suite, but that may miss
scenarios that only occur in optimized builds or that are not represented
by the test suite.

%%% Local Variables: 
%%% mode: latex
%%% TeX-master: "paper"
%%% End: 
