%!TEX root = paper.tex
%
\begin{abstract}
The Servo project is an effort started at Mozilla Research to build a new web browser
engine that takes advantage of both modern parallel hardware and advances in language
technology that prevent many of the common security errors that plague all of the other
modern browsers.
The new language, Rust, enabled us to take on this project in a type-centric way instead of
the traditional C++ that relies on coding conventions, imprecise static analyses, and
fuzzers.
While many languages before have offered some combination of memory safety, parallelism,
and concurrency, Rust is, to our knowledge, the first to do so that provides the programmer
with the control over both memory and generated code that is required by systems programmers.
In this paper, \emph{systems} \emph{software} refers to programs that have some combination
of soft realtime requirements and memory, CPU, and power usage constraints that require the
programmer to be able to predict these features from their source code --- without being a
compiler developer.

Servo is currently over 300k lines of code and implements enough of the web to render and
process many pages, though it is still a far cry from the over 7 million lines of code in
the Mozilla Firefox browser and its associated libraries.
However, we believe that we have implemented enough of the web platform to provide an
early report on the successes, failures, and open problems remaining in Servo, from the
point of view of programming languages and runtime research.
It is our hope that this work will spur additional research in new language features
suitable for a systems language that also scale up to projects the size of a web browser
engine.
\end{abstract}
