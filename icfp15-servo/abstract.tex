%!TEX root = paper.tex
%
\begin{abstract}
All modern web browsers --- Internet Explorer, Firefox, Chrome, Opera, and Safari ---
have a core rendering engine written in \Cplusplus{}.
This language choice was made because it affords the systems programmer complete control
of the underlying hardware features and memory in use, and it provides a transparent
compilation model.

Servo is a project started at Mozilla Research to build a new web browser engine that
preserves the capabilities of these other browser engines but also both takes 
advantage of the recent trends in parallel hardware and is more memory-safe.
We use a new language, Rust, that provides us a similar level of control of the underlying
system to \Cplusplus{} but which builds on many concepts familiar to the functional programming
community, forming a novelty --- a useful, safe systems programming language.

In this paper, we show how a language with an affine type system, ownership types,
regions, and many syntactic features familiar to functional language programmers can be
successfully used to build state-of-the-art systems software.
We also outline several pitfalls encountered along the way and describe some potential
areas for future research.
\end{abstract}
