%!TEX root = paper.tex
\section{Introduction}
\label{sec:intro}
The heart of a modern web browser is the browser engine, which is the code responsible
for loading, processing, evaluating, and rendering web content.
There are three current browser engine families:
\begin{enumerate}
\item Trident/Spartan, the engine in Internet Explorer~\cite{IE}
\item Webkit\cite{WEBKIT}/Blink, the engine in Safari~\cite{SAFARI}, Chrome~\cite{CHROME}, and Opera~\cite{OPERA}
\item Gecko, the engine in Firefox~\cite{FIREFOX}
\end{enumerate}
All of these engines have at their core many millions of lines of \Cplusplus{} code.
While the use of \Cplusplus{} has enabled all of these browsers to achieve excellent sequential
performance on a single web page, on mobile devices with lower processor speed but many
more processors, these browsers do not provide the same level of interactivity that they
do on desktop processors~\cite{parallelizing-web-pages,ZOOMM}.
Further, in an informal inspection of the critical security bugs in Gecko, we determined that
roughly 50\% of the bugs are use after free, out of range access, or related to integer
overflow.
The other 50\% is split between errors in tracing values from the JavaScript heap in the
\Cplusplus{} code and errors related to dynamically compiled code.

Servo~\cite{SERVO} is a new web browser engine designed to address the major environment and 
architectural changes over the last decade.
The goal of the Servo project is to produce a browser that enables new applications to be authored
against the web platform that run with more safety, better performance, and better power usage
than in current browsers.
To address memory-related safety issues, we are using a new systems programming language,
Rust~\cite{RUST}.
For parallelism and power, we scale across a wide variety of hardware by building either data-
or task-parallelism, as appropriate, into each part of the web platform.
Additionally, we are improving concurrency by reducing the simultaneous access to data
structures and using a message-passing architecture between components such as the
JavaScript engine and the rendering engine that paints graphics to the screen.
Servo is currently over 300k lines of code and implements enough of the web to render and
process many pages, though it is still a far cry from the over 7 million lines of code in
the Mozilla Firefox browser and its associated libraries.
However, we believe that we have implemented enough of the web platform to provide an
early report on the successes, failures, and open problems remaining in Servo, from the
point of view of programming languages and runtime research.
In this experience report, we discuss the design and architecture of a modern web 
browser engine, show how modern programming language techniques --- many of which
originated in the functional programming community --- address these design 
constraints, and also touch on ongoing challenges and areas of research where we
would welcome additional community input.

%%% Local Variables: 
%%% mode: latex
%%% TeX-master: "paper"
%%% End: 
