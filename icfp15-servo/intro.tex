%!TEX root = paper.tex

\section{Introduction}
\label{sec:intro}
Servo\footnote{https://github.com/servo/servo} is a new web browser engine whose development
was started in February of 2013.

Modern web browser engines have a similar complexity level to that of operating systems.
These engines are expected to execute arbitrary user programs within a safe sandbox, scale 
across a wide variety of hardware, manage access to shared system resources, and further 
do all of that while providing good performance and correct behavior
against sometimes-incomplete standards.
With the expectation of executing well on mobile devices and parallel hardware, the 
fundamental design assumptions have significantly changed since all of the modern
web engines were originally designed.

Servo is a new web browser engine designed to address the major environment and 
architectural changes over the last decade.
To address security, we are using a new systems programming language, Rust,
which adds static type checking to address by far the top cause of security
issues in modern web browsers --- failure in handling manual memory allocation
in \Cplusplus{}.
We scale across a wide variety of hardware by building either data- or task-parallelism, 
as appropriate, to each of the stages of the Web layout pipeline.
Finally, we are improving concurrency by reducing the simultaneous access to data
structures and using a message-passing architecture between components such as the
JavaScript engine and the rendering engine that is painting graphics to the screen.
In this experience report, we discuss the design and architecture of a modern web 
browser engine, show how modern programming language techniques --- many of which
originated in the functional programming community --- address these design 
constraints, and also touch on ongoing challenges and areas of research where we
would welcome additional community input.

%%% Local Variables: 
%%% mode: latex
%%% TeX-master: "paper"
%%% End: 
