%!TEX root = paper.tex

\section{Rust}
\label{sec:rust}

Rust is a statically typed systems programming language most heavily inspired by the C
and ML families of languages~\cite{RUST}.
Like the C family of languages, it provides the developer fine control over memory layout
and predictable performance.
Unlike C programs, Rust programs are \emph{memory safe} by default,
only allowing unsafe operations in specially-delineated blocks.

Rust features an \emph{ownership-based} type system inspired by the region systems work in the
MLKit project~\cite{mlkit} and especially as implemented in the Cyclone language~\cite{cyclone}.
Unlike the related ownership system in Singularity OS~\cite{singularity}, Rust allows programs to
not only transfer ownership but also to temporarily \emph{borrow} owned values, significantly
reducing the number of region and ownership annotations required in large programs.
The ownership model encourages immutability by default while allowing for controlled
mutation of owned or uniquely-borrowed values.

Complete documentation and a language reference for Rust are available at: \url{http://doc.rust-lang.org/}.

\subsection{Ownership and concurrency}
Because the Rust type system provides very strong guarantees about memory aliasing,
Rust code is memory safe even in concurrent and multithreaded environments,
but beyond that Rust also ensures \emph{data-race freedom}.

In concurrent programs, the data operated on by distinct threads is also itself distinct:
under Rust's ownership model, data cannot be owned by two threads at the same time.
For example, the code in \figref{fig:bad-concurrency} generates a static error from the compiler because
after the first thread is spawned, the ownership of \lstinline{data} has been transferred into the
closure associated with that thread and is no longer available in the original thread.
%http://is.gd/tkcD9R
\begin{figure}
\begin{lstlisting}
fn main() {
  // An owned pointer to a heap-allocated
  // integer
  let mut data = Box::new(0);

  Thread::spawn(move || {
    *data = *data + 1;
  });
  // error: accessing moved value
  print!("{}", data);
}
\end{lstlisting}
  \caption{Code that will not compile because it attempts to access mutable state from two threads.}
  \label{fig:bad-concurrency}
\end{figure}

On the other hand, the immutable value in \figref{fig:shared-concurrency} can be borrowed and shared between multiple threads as long as those threads don't outlive the scope of the data, and even mutable values can be shared as long
as they are owned by a type that preserves the invariant that mutable memory is unaliased, as with the
mutex is \figref{fig:shared-mutable-concurrency}.

\begin{figure}
\begin{lstlisting}
fn main() {
  // An immutable borrowed pointer to a
  // stack-allocated integer
  let data = &1;

  Thread::scoped(|| {
    println!("{}", data);
  });
  print!("{}", data);
}
\end{lstlisting}
  \caption{Safely reading immutable state from two threads.}
  \label{fig:shared-concurrency}
\end{figure}

\begin{figure}
\begin{lstlisting}
fn main() {
  // A heap allocated integer protected by an
  // atomically-reference-counted mutex
  let data = Arc::new(Mutex::new(0));
  let data2 = data.clone();

  Thread::scoped(move || {
    *data2.lock().unwrap() = 1;
  });

  print!("{}", *data.lock().unwrap());
}
\end{lstlisting}
  \caption{Safely mutating state from two threads.}
  \label{fig:shared-mutable-concurrency}
\end{figure}

With relatively few simple rules, ownership in Rust enables foolproof task parallelism,
but also data parallelism, by partitioning vectors and lending mutable references into properly scoped threads.
Rust's concurrency abstractions are entirely implemented in libraries, and though
many advanced concurrent patterns such as work-stealing~~\cite{blumeofe:multiprogrammed-work-stealing}
cannot be implemented in safe Rust, they can usually be encapsulated in a memory-safe interface.

%%% Local Variables: 
%%% mode: latex
%%% TeX-master: "paper"
%%% End: 
